\usetikzlibrary{shapes,arrows,mmt}
\def\thmo#1#2{\mathsf{#1}\colon\kern-.15em{#2}}
\providecommand\myxscale{3.9}
\providecommand\myyscale{2.2}
\providecommand\myfontsize{\footnotesize}

\begin{figure}[h]
  \begin{center}
    \begin{tikzpicture}[xscale=\myxscale,yscale=\myyscale]\myfontsize

      %Background
      \draw[rounded corners, draw=none, fill=lightgray, opacity=0.5] (-1.7,-3.5) -- (-1.7,-0.5) -- (0.4,-0.5) -- (0.4,-3.5) -- cycle;

      % Theory of Dimensions
      \node[thy] (dim) at (-1,-3) {
        \begin{tabular}{l}
          \textsf{Dimensions}\\\hline
          8 basic dimensions\\
          dimension multiplication\\
          dimension division\\
          \hline
        \end{tabular}
      };

      % Dimensions extended
      \node[thy] (dimex) at (-1,-2) {
        \begin{tabular}{l}
          \textsf{Dimensions Extended}\\\hline
          aliases for composite dimensions\\
          (such as $\mathsf{area} = \mathsf{length} \cdot{} \mathsf{length}$)\\
          \hline
        \end{tabular}
      };

      \draw[include] (dim) -- (dimex);

      %Numbers
      \node[thy] (numbers) at (0,-2) {
        \begin{tabular}{l}
          \textsf{Numbers}\\\hline
          basic numbers\\
          (OpenMath)\\
          \hline
        \end{tabular}
      };

      % Quantity Expressions
      \node[thy] (qes) at (-0.75,-1) {
        \begin{tabular}{l}
          \textsf{Quantity Expressions}\\\hline
          quantity expression constructors\\
          quantity expression multiplication\\
          quantity expression division\\
          \hline
        \end{tabular}
      };

      \draw[include] (numbers) -- (qes);
      \draw[include] (dimex) -- (qes);

      %SI
      \node[thy] (SI) at (-1,0) {
        \begin{tabular}{l}
          \textsf{SI}\\\hline
          basic SI units for all dimensions\\
          \hline
        \end{tabular}
      };

      \draw[include] (qes) -- (SI);

      %Imperial Lengths 1A
      \node[thy] (ImpA1) at (0.3,0) {
        \begin{tabular}{l}
          \textsf{Imperial Lengths A1}\\\hline
          Thou\\
          \hline
        \end{tabular}
      };

      \draw[include] (qes) -- (ImpA1);
      \draw[view] (ImpA1) -- (SI);


      %Imperial Lengths A2
      \node[thy] (ImpA2) at (1.5,0) {
        \begin{tabular}{l}
          \textsf{Imperial Lengths A2}\\\hline
          Inch, Foot, Yard, \\
          Chain, Fathom, Mile, \\
          Mile, Furlong, Cable, \\
          Nautical Mile, League\\
          \hline
        \end{tabular}
      };

      \draw[include] (ImpA1) -- (ImpA2);

      %Imperial Lengths B1
      \node[thy] (ImpB1) at (0.9,-1) {
        \begin{tabular}{l}
          \textsf{Imperial Lengths B1}\\\hline
          Link\\
          \hline
        \end{tabular}
      };

      \draw[include] (qes) -- (ImpB1);
      \draw[view] (ImpB1) -- (ImpA2);

      %Imperial Lengths B2
      \node[thy] (ImpB2) at (1,-2) {
        \begin{tabular}{l}
          \textsf{Imperial Lengths B2}\\\hline
          Rod\\
          \hline
        \end{tabular}
      };

      \draw[include] (ImpB1) -- (ImpB2);


      % Imperial Lengths
      \node[thy] (ImpLen) at (1.75,-1) {
        \begin{tabular}{l}
          \textsf{Imperial Lengths}\\\hline
        \end{tabular}
      };

      \draw[include] (ImpA2) -- (ImpLen);
      \draw[include] (ImpB2) -- (ImpLen);

      % Imperial Area
      \node[thy] (ImpArea) at (1.75,-2.7) {
        \begin{tabular}{l}
          \textsf{Imperial Area}\\\hline
          Perch, Rood, Acre\\\hline
        \end{tabular}
      };

      \draw[include] (ImpLen) -- (ImpArea);

    \end{tikzpicture}
  \end{center}

  \caption[A Small Part Of The Graph Containing Unit Theories. ]{A Small Part Of The Graph Containing Unit Theories\footnote{Imports are represented as solid edges and views as wavy edges. The gray area contains basic theories that are used to define quantity expressions. }. }
  \label{fig:unitsgraph}
\end{figure}

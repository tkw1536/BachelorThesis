\section{Introduction}

\subsection{The Problem Of Units}

Units are everywhere. We encounter units and quantity expressions in everyday life wherever we go. When driving on the road we see a speed limits on signs (for example $30 \frac{\text{km}}{s}$). When we go shopping there are different shoe sizes. When we buy something, we pay a currency of $30 \text{\euro}$. Everything is being quantified. This is also the case in science papers. Many, if not all, papers have 1 or more quantity expressions in them. Approximatly 1\% of all letters in scienficic papers belong to a quantity expression\ednote{Verify this}.

This in itself is not a problem. The problem occurs when different units are used to describe the same quantities. In most of the world, the metric system is used to describe most quantities. Some countries however still use non-SI units which often leads to accidents.

These different units sometimes make it very difficult to talk about Quantity Expressions. One notable example for this is the \textit{Mars Climate Orbiter} which was destroyed in 1999 when it entered the atmosphere of Mars because it received the non-standard units of \textit{pound-seconds} instead of the expected \textit{newton-seconds}\cite{nasa:mcor}.

\subsection{State Of The Art: Unit Converters And The SI System}

Out of convenience new units ones are constantly being invented. There many hundred units that already exist. Just converting between them is easy, googleing ``unit converter'' reveals about 12000000 results\ednote{Perhaps quote the google result page here?}. Even Google itself has implemented a unit converter into its search engine.

However, all of these tools only convert units directly. You can enter the original quantity and choose the target units. There is no tool which directly incorperates the translation of units into search results by finding results even if they are in a different form. This would make search efforts much easier, as there would no longer be a need for direct conversion from the user. Furthermore unit converters are commonly very restricted in the units they support. They do not have a way to easily expand the system of units.

This handling of units is very superficial and is not actually taking the underlying meaning of the quantity expression into account. The SI specification \cite{sispec} on the other hand provides a very good insight into how units can be handled. This very formal approach does not take every special situation into account so it is impractical to use SI units for everything. Neither this nor the previous approach can thus easily be used. Both require a lot of user interaction.

\subsection{Our Approach: A Semantic Quantity Expression Search}

That is why we want to build a search engine that unifies these approaches. It should be capable to find occurrences of quantity expressions within documents, no matter in which form they are written. For this we need several components: (1) an extensible system flexible enough to convert between units when needed, (2) a so-called spotter that finds occurrences of quantity expressions within documents, (3) a search algorithm that can take a quantity expression from the user and find equivalent ones in the results from the spotter and (4) a front-end that allows the user to enter a quantity expression and receive the results.

For (1) we want to use a meta-mathematical theories approach. With the help of MMT this allows us to build an extendible unit system. This uses a concept of a theory graph in which units are related via so-called views and imports. These can be used to translate between them. Additionally, we can define a new unit and easily link it to any of the ones which have been defined previously. Point (2) will be taken on by Stiv Sherko in a seperate effort \cite{proposal:sharko}. The spotter finds quantity expressions inside documents which can almost directly be used by our system. For our search algorithm (3) we use a simple trick: When finding units, we normalise them to a normal form. This normal form is then used to efficiently index the harvest delivered from the previous step. Finally for (4) we built a frontend which allows the user to enter quantity expressions. It is deployed at \ednote{Link the deploy site here. }.

\ednote{Make a system diagram and refer to it here. }

\subsection{Overview}

This thesis is organised as follows: In section \ref{sec:mathoverview} we start by giving an introduction to mathematical theory modeling. We proceed in section \ref{sec:strucqe} to talk about how quantity expressions can be formalised and in section \ref{sec:mqes} we apply these insights in order to start building in our search engine. In section \ref{sec:pit} we present the implementation we designed and explain its advantages and disadvantages. After discussing future work in section \ref{sec:future} we conclude in section \ref{sec:conclusion}.

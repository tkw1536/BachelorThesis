\section{Introduction}

Quantity


%In this thesis we want to give an approach to Semantic Search for Quantity Expressions. A quantity expression is a scalar together with a physical unit, for example $25 \frac{\text{m}}{\text{s}}$ where $25$ is the scalar and $\frac{\text{m}}{\text{s}}$ the unit. This quantity expression is semantically equivalent to $90 \frac{\text{km}}{\text{h}}$ (with equivalent in this sense meaning it expresses the same quantity) although the units are different. In a semantic search for quantity expressions we want to be able to search for a certain quantity expression and find any equivalent ones as well.

%MathWebSearch (MWS for short) is an existing semantics-aware system to search \LaTeX \  documents \footnote{Technically, MWS itself can only search XHTML documents. However with the help of \LaTeX{}ML \cite{Miller:latexml:base} it also handles \LaTeX \ documents. } for mathematical formulae \cite{HamKohPro:man14}. As it is semantics-aware it not only searches for formulae in a simple-minded ``text search'' way but it can deal with wildcards such as $ { \color{red} x} + \sqrt{\color{red} x}$. For this query MWS would deliver forumlae as above where $ { \color{red} x} $ has been substituted with any sub-formula. In this way MWS abstracts from variable names.

%We want to design a search engine that is similar to the existing MathWebSearch system in that is abstracts from the form of the quantity expression. This system should be very flexible with respect to the units it supports. We want a system in which it is easy to add new units. Eventually we also want to integrate our system into MathWebSearch so that we do not have to re-write existing search algorithms and can also use wildcards as described above.

This thesis is organised as follows: In section \ref{sec:mathoverview} we start by giving an introduction to mathmatical theory modeling. We proceed in section \ref{sec:strucqe} to talk about how quantity expressions can be formalised and in section \ref{sec:mqes} we apply these insights in order to start building in our search engine. In section \ref{sec:pit} we present the implementation we designed and discuss its advantages and disadvantages. After discussing future work in section \ref{sec:future} we conclude in section \ref{sec:conclusion}. 

\section{The structure of Quantity Expressions}

The first step in developing a good search engine for Quantity Expressions is to take a closer look at Quantity Expression.

\subsection{Compositional behaviour of Quantity Expressions}

For this purpose let us take a look at:
\[x = 25\ \frac{\text{m}}{\text{s}}\]
We notice that $x$ consists of 2 parts, a scalar ($25$) and a scalar-free $\frac{\text{m}}{\text{s}}$. Furthermore, the unit consists of two primitive units $m$ and $s$. Since they are divided by one another, we can conclude that $x$ describes a velocity.

While $m$ and $s$ are certainly primitive units (they can not be decomposed further), it is not easy to define a unit as a composition of simple units. Consider the following example:
\[y = \frac{\text{L}}{100\ \text{km}}\]
$y$ is certainly a unit and also a quantity expression. It consists of 2 sub-expressions, $\text{L}$ and $100\ \text{km}$. The first one is a primitive unit and the second one a multiplication of a number and the unit $\text{km}$. It is thus reasonable to define the following 5 types of quantity expressions:
\begin{enumerate}
  \item A primitive unit, such as $\text{m}$ (meter). This is the most obvious one.
  \item the Multiplication of a quantity expression with a scalar.
  \item the Division of a quantity expression by a scalar.
  \item The multiplication $\cdot{}$ which takes 2 existing quantity expression and generates a new one, for example $\cdot \left(100, \text{m} \right) = 100\ \text{m}$
  \item The division \textbackslash which again takes 2 quantity expressions and generates a new one, for example $\text{\textbackslash} (\text{m}, \text{s}) = \frac{\text{m}}{\text{s}}$
\end{enumerate}
This allows us to easily generate the quantity expressions $x$ and $y$ from primitive units m, s, L and km.

\subsection{Dimensions of Quantity Expressions}
Now let us briefly examine the dimensions of Quantity Expressions. The dimension of a quantity expression is the type of quantity it expresses. For example $5\ \text{m}$ describes some length. According to the International System of Units\ednote{Quote something properly here} there are seven basic dimensions:
\begin{enumerate}
  \item length
  \item mass
  \item time
  \item electric current
  \item temperature
  \item luminous intensity
  \item amount of substance.
\end{enumerate}

In addition to these dimensions there are 2 more special dimensions: The \textit{count} dimension (used for counting of objects) and the \textit{none} dimension for dimensionless quantities. \ednote{Explain these more}

Similar to the compositional behaviour of quantity expressions, dimensions can be multiplied and divided. Unlike quantity expressions however they can not be multiplied with numbers. Furthermore, when multiplying to quantity expressions of dimensions $a$ and $b$ their resulting dimension is $a \cdot{} b$, the multiplication of the dimensions. The same goes for division. In this regard the dimension of a quantity expression behaves like a type.

\subsection{Mathematical Theory of Quantity Expressions}
\label{sec:qeform}

The realisation that dimensions are types of quantity expressions leads us to our first formalisation of quantity expressions. We first define a theory of dimensions in figure \ref{fig:dimensions} and then import it to define a theory of Quantity Expressions in figure \ref{fig:QE}.

\begin{figure}[h]
  \begin{center}
    \begin{tabular}{|l c l|}
      \hline
      \textsf{Dimension} & &\\\hline
      $\mathsf{dimension}$ & $:$ & $ \mathsf{type}$\\

      $\mathsf{none}$ & $:$ & $ \mathsf{dimension}$\\
      $\mathsf{count}$ & $:$ & $ \mathsf{dimension}$\\
      $\mathsf{length}$ & $:$ & $ \mathsf{dimension}$\\
      $\mathsf{mass}$ & $:$ & $ \mathsf{dimension}$\\
      $\mathsf{time}$ & $:$ & $ \mathsf{dimension}$\\
      $\mathsf{current}$ & $:$ & $ \mathsf{dimension}$\\
      $\mathsf{temperature}$ & $:$ & $ \mathsf{dimension}$\\
      $\mathsf{luminous}$ & $:$ & $ \mathsf{dimension}$\\
      $\mathsf{amount}$ & $:$ & $ \mathsf{dimension}$\\

      $\cdot{}$ & $:$ & $ \mathsf{dimension} \rightarrow \mathsf{dimension} \rightarrow \mathsf{dimension}$\\
      $/$ & $:$ & $ \mathsf{dimension} \rightarrow \mathsf{dimension} \rightarrow \mathsf{dimension}$\\\hline
    \end{tabular}
  \end{center}
  \caption{A Formalization Of The Theory Of Dimensions. }
  \label{fig:dimensions}
\end{figure}


\begin{figure}[h]
  \begin{center}
    \begin{tabular}{|l c l|}
      \hline
      \textsf{Quantity Expression} & &\\\hline
      $ \mathsf{import \ Dimension}$ &&\\
      $ \mathsf{import \ Number}$ &&\\
      \hline
      $\mathsf{QE}$ & $:$ & $ \mathsf{dimension} \rightarrow \mathsf{type}$\\
      $\mathsf{QENMul}$& $:$ & $ \forall x : \mathsf{dimension} . \mathsf{number} \rightarrow \mathsf{QE}\left( x\right) \rightarrow \mathsf{QE}\left( x\right)$\\
      $\mathsf{QENDiv}$& $:$ & $ \forall x : \mathsf{dimension} . \mathsf{QE}\left( x\right) \rightarrow \mathsf{number} \rightarrow \mathsf{QE}\left( x\right)$\\

      $\mathsf{QEMul}$& $:$ & $ \forall x : \mathsf{dimension} . \forall y : \mathsf{dimension} . \mathsf{QE}\left( x\right) \rightarrow \mathsf{QE}\left( y\right) \rightarrow \mathsf{QE} \left( \cdot{} \left(x, y\right) \right)  $\\
      $\mathsf{QEAdd}$& $:$ & $ \forall x : \mathsf{dimension} . \mathsf{QE}\left( x\right) \rightarrow \mathsf{QE}\left( x\right) \rightarrow \mathsf{QE} \left( x \right)  $\\
      $ \mathsf{QEDiv}$& $:$ & $ \forall x : \mathsf{dimension} . \forall y : \mathsf{dimension} . \mathsf{QE}\left( x\right) \rightarrow \mathsf{QE}\left( y\right) \rightarrow \mathsf{QE} \left( \backslash \left(x, y\right) \right)  $\\\hline
    \end{tabular}
  \end{center}

  \caption{The Chosen Formalization Of The Theory Of Quantity Expressions. }
  \label{fig:QE}
\end{figure}


Figure \ref{fig:dimensions} defines the 9 basic dimensions and then dimension compositon via multiplication and division. Then we move on in \ref{fig:QE} to define quantity expressions. Each quantity expression has a dimension (via the \textit{QE} constant). This allows us to define basic units (which we will actually do in the next figure. ). With the \textit{QENMul} and \textit{QENDiv} symbols we can multiply and divide quantity expressions by numbers ( for this case we actually need to import some theory of numbers to allow us actually write this down as a Term). Then we define multiplication and division of quantity expressions in such a way that dimensions multiply and divide appropriatly.

Now we need to introduce some basic units. Let us start by just defining meter in figure \ref{fig:meter}. We can now write a term in this theory that expresses any number of meters.

\begin{figure}[h]
  \begin{center}
    \begin{tabular}{|l c l|}
      \hline
      \textsf{Meter} & &\\\hline
      $ \mathsf{import \ Quantity Expression}$ &&\\
      \hline
      $\mathsf{Meter}$ & $:$ & $ \mathsf{QE} \left( \mathsf{length} \right)$\\\hline
    \end{tabular}
  \end{center}

  \caption{A Theory Defining The Primitive Unit Meter. }
  \label{fig:meter}
\end{figure}


\subsection{Transforming Quantity Expressions from one form into another}
\label{sec:qeconv}

This is a very nice start of a unit system\ednote{Better formulation}. Let us know define a few more units of length. In figure \ref{fig:nonsilength} we show a few non-si units. Here we first define thoug as a quanity expression of length and then one-by-one define more units in terms of the previous one.
\begin{figure}[h]
  \begin{center}
    \begin{tabular}{|l c l|}
      \hline
      \textsf{Meter} & &\\\hline
      $ \mathsf{import \ Quantity Expression}$ &&\\
      \hline
      $\mathsf{Thou}$ & $:$ & $ \mathsf{QE}\left( \mathsf{length} \right)$\\
      $\mathsf{Foot}$ & $:$ & $ \mathsf{QE}\left( \mathsf{length} \right) = \mathsf{QEMul} \left( \mathsf{QENum} \left(1000 \right), \mathsf{Thou} \right)$\\
      $\mathsf{Yard}$ & $:$ & $ \mathsf{QE}\left( \mathsf{length} \right) = \mathsf{QEMul} \left( \mathsf{QENum} \left(3 \right), \mathsf{Foot} \right)$\\
      $\mathsf{Chain}$ & $:$ & $ \mathsf{QE}\left( \mathsf{length} \right) = \mathsf{QEMul} \left( \mathsf{QENum} \left(22 \right), \mathsf{Yard} \right)$\\
      $\mathsf{Furlong}$ & $:$ & $ \mathsf{QE}\left( \mathsf{length} \right) = \mathsf{QEMul} \left( \mathsf{QENum} \left(10 \right), \mathsf{Chain} \right)$\\
      $\mathsf{Mile}$ & $:$ & $ \mathsf{QE}\left( \mathsf{length} \right) = \mathsf{QEMul} \left( \mathsf{QENum} \left(8 \right), \mathsf{Furlong} \right)$\\
      \hline
    \end{tabular}
  \end{center}

  \caption{A theory of some non-SI units of length. }
  \label{fig:nonsilength}
\end{figure}


We now want to relate quantity expressions with units from the \textit{Meter} theory to units from the \textit{Non SI Lengths} theory. It is known that $1 \mathsf{Thou} = 0.0000254 \mathsf{Meter}$. This can be easily expressed with a view $\psi$ between these to theories:

\[
\psi = \left\{\begin{array}{l}
  \mathsf{Thou} \mapsto \mathsf{QENMul} \left( 0.0000254, \mathsf{Meter} \right)
\end{array}\right\}
\]

Even though the view just maps the symbol \textit{Thou} to some Term in the Meter theory, we can also use it to transform any other term from the Thou Theory. Since all units are defined in terms of the previous one, we can just expand all definitions to get an expression containing only numbers and the unit \textit{Thou}. Then we can use the view as normal to get a Quantity Expression in the \textit{Meter} theory.

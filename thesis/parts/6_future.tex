\section{Current Limits And Future work}
\label{sec:future}

While the current implementation provides a proof of concept that the choosen approach indeed works, there are several problems with it. Furthermore there are several features\ednote{Better word} that the current system can not yet do. \ednote{Expand paragraph}
\ednote{Maybe re-order the subsections}.

\subsection{Absolute Vs Relative Units}
The current unit system is easily extensible and can express a lot of different units easily.

However there also is a major limitation that stops it from being able to translate between \textit{Kelvin} and \textit{Celsius} as Units of Temperature. This is due to the fact that \textit{Views} only map \textit{Constants} directly. Due to the way we have formalised Quanrtity Expressions this only allows \textit{linear maps} as translations between units.

In \cite{SD:UnitKnowledgeMgmt08} Davenport distinguishes between two types of units: Absolute and relative\footnote{In the paper these are simply called non-absolute units. }. When adding two Quantity Expressions with the same unit we can usually just factor out the unit and then add up the scalars, in other words the set of these QEs forms an abelian monoid. He discovered that this is not always the case. We have $2\ \text{Kelvin} + 3\ \text{Kelvin} = 5\ \text{Kelvin}$ but $2\ \text{Celsius} + 3\ \text{Celsius} \neq 5\ \text{Celsius}$. \textit{Celsius} is not defined via a single quantity expression but is just $\frac{1}{100}$ of the difference between two other Quantity Expressions. Hence it is not an absolutely defined unit, but rather a \textit{relative} unit.

Our implementation can only handle \textit{absolute units}. A possible solution to this problem is to choose a different formalisation of units. Currently they are just some quantity expressions. If we considder a unit as a map from \textit{Real numbers} to \textit{Quantities of a certain Dimension} we can easily translate them in a relative fashion using (lambda) functions:
\[\text{Celsius}(x) := \textit{Kelvin}(x - 273.5)\]
Such a change would obviously require changes in the way the normalisation currently works.

\subsection{Handling of SI Prefixes}
In the current

* type equalities
* requirements of the graph
* (in)stabilities of MMT
* davenport units (relative units)

\subsection{Extension Of The Theory Graph Of Units}
\ednote{Write this section}
\subsection{Crude state of the results page}
\label{sec:fut_res}
\subsection{Integration With MathWebSearch}
\ednote{Write this section}
\subsection{Enhancement of the Spotter}
\ednote{Write this section}

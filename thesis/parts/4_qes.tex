\section{Making Quantity Expressions searchable}
Now that we know what quantity expressions are and how we can convert between equivalent representations, it is time to start about concrete algorithms for our search engine.

When querying this search engine with a quantity expression, we want to be able find occurrences of equivalent quantity expressions. For this we need several components:
\begin{enumerate}
  \item a component that allow the user to enter quantity expressions,
  \item a way to send quantity expressions from the user to the search engine
  \item a software to find quantity expressions within documents
  \item an algorithm to convert between different forms of a quantity expressions (or at least determine if 2 quantity expressions are equivalent) and finally
  \item a unit system that is flexible enough to handle different kinds of units.
\end{enumerate}

\subsection{An idea for an extensible system}
\label{sec:meq_model}

Let us start with point (5), a unit system that is flexible enough to handle all different of units. As a basis we use the formalisation described in section \ref{sec:qeform}. Since we want to convert different units, we also want to apply what we learned in section \ref{sec:qeconv}. We use MMT to write down the formalisation of quantity expressions as described. We can write down a lot of different units in this form. In figure \ref{fig:unitsgraph} you can see a small subgraph of unit theories.

\usetikzlibrary{shapes,arrows,mmt}
\def\thmo#1#2{\mathsf{#1}\colon\kern-.15em{#2}}
\providecommand\myxscale{3.9}
\providecommand\myyscale{2.2}
\providecommand\myfontsize{\footnotesize}

\begin{figure}[h]
  \begin{center}
    \begin{tikzpicture}[xscale=\myxscale,yscale=\myyscale]\myfontsize

      %Background
      \draw[rounded corners, draw=none, fill=lightgray, opacity=0.5] (-1.7,-3.5) -- (-1.7,-0.5) -- (0.4,-0.5) -- (0.4,-3.5) -- cycle;

      % Theory of Dimensions
      \node[thy] (dim) at (-1,-3) {
        \begin{tabular}{l}
          \textsf{Dimensions}\\\hline
          8 basic dimensions\\
          dimension multiplication\\
          dimension division\\
          \hline
        \end{tabular}
      };

      % Dimensions extended
      \node[thy] (dimex) at (-1,-2) {
        \begin{tabular}{l}
          \textsf{Dimensions Extended}\\\hline
          aliases for composite dimensions\\
          (such as $\mathsf{area} = \mathsf{length} \cdot{} \mathsf{length}$)\\
          \hline
        \end{tabular}
      };

      \draw[include] (dim) -- (dimex);

      %Numbers
      \node[thy] (numbers) at (0,-2) {
        \begin{tabular}{l}
          \textsf{Numbers}\\\hline
          basic numbers\\
          (OpenMath)\\
          \hline
        \end{tabular}
      };

      % Quantity Expressions
      \node[thy] (qes) at (-0.75,-1) {
        \begin{tabular}{l}
          \textsf{Quantity Expressions}\\\hline
          quantity expression constructors\\
          quantity expression multiplication\\
          quantity expression division\\
          \hline
        \end{tabular}
      };

      \draw[include] (numbers) -- (qes);
      \draw[include] (dimex) -- (qes);

      %SI
      \node[thy] (SI) at (-1,0) {
        \begin{tabular}{l}
          \textsf{SI}\\\hline
          basic SI units for all dimensions\\
          \hline
        \end{tabular}
      };

      \draw[include] (qes) -- (SI);

      %Imperial Lengths 1A
      \node[thy] (ImpA1) at (0.3,0) {
        \begin{tabular}{l}
          \textsf{Imperial Lengths A1}\\\hline
          Thou\\
          \hline
        \end{tabular}
      };

      \draw[include] (qes) -- (ImpA1);
      \draw[view] (ImpA1) -- (SI);


      %Imperial Lengths A2
      \node[thy] (ImpA2) at (1.5,0) {
        \begin{tabular}{l}
          \textsf{Imperial Lengths A2}\\\hline
          Inch, Foot, Yard, \\
          Chain, Fathom, Mile, \\
          Mile, Furlong, Cable, \\
          Nautical Mile, League\\
          \hline
        \end{tabular}
      };

      \draw[include] (ImpA1) -- (ImpA2);

      %Imperial Lengths B1
      \node[thy] (ImpB1) at (0.9,-1) {
        \begin{tabular}{l}
          \textsf{Imperial Lengths B1}\\\hline
          Link\\
          \hline
        \end{tabular}
      };

      \draw[include] (qes) -- (ImpB1);
      \draw[view] (ImpB1) -- (ImpA2);

      %Imperial Lengths B2
      \node[thy] (ImpB2) at (1,-2) {
        \begin{tabular}{l}
          \textsf{Imperial Lengths B2}\\\hline
          Rod\\
          \hline
        \end{tabular}
      };

      \draw[include] (ImpB1) -- (ImpB2);


      % Imperial Lengths
      \node[thy] (ImpLen) at (1.75,-1) {
        \begin{tabular}{l}
          \textsf{Imperial Lengths}\\\hline
        \end{tabular}
      };

      \draw[include] (ImpA2) -- (ImpLen);
      \draw[include] (ImpB2) -- (ImpLen);

      % Imperial Area
      \node[thy] (ImpArea) at (1.75,-2.7) {
        \begin{tabular}{l}
          \textsf{Imperial Area}\\\hline
          Perch, Rood, Acre\\\hline
        \end{tabular}
      };

      \draw[include] (ImpLen) -- (ImpArea);

    \end{tikzpicture}
  \end{center}

  \caption{A small part of the graph containing unit theories. Imports are represented as solid edges and views as squiged edges. The gray area contains basic theories that are used to define quantity expressions. }
  \label{fig:unitsgraph}
\end{figure}


\ednote{More}

\subsection{Normalisation of Quantity Expressions}

Now that we have a basic formalisation of units, we want to be able to exchange quantity expressions between the user, the search engine and the


\subsection{Serialising Quantity Expressions to XML}
* W3C RFC, find it again

%\label{sec:mws:tg}

%The theories and their relations can be represented as a graph, called a theory graph. In Figure \ref{graph1} we can see a simple example of a theory graph.

%\input{graph1.tex} \ednote{Explain the mappings somewhere}

%Every theory and every constant in MMT has a (globally unique) URI, called the MMT URI. It can be constructed via a triple $\left( G,M,S \right)$ where $G$ is a document URI, $M$ is a module name within this document and $S$ is the name of a constant \cite{RabKoh:WSMSML13}. We then seperate these components via a ? to get a URI of the form $G?M?S$. Because every theorem in MMT is only a symbol declaration it can also be represented via a URI. As explained above, if two theories are related via a view or import, theorems can be translated along this relation.

%Sometimes it is useful not to write down a theorem explicitly, but only give an existing theorem and translate this along a view or import. The MMT URI of this induced theorem can then be given using the view. This URI contains enough information for MMT to generate the theorem in explicit form\cite{IanKohProd:rassmk14}.

%In an MWS Harvest we exploit this property of induced theorems. We have several represented theorems, the forumlae in the corpus, and many more induced theorems, other equivalent representations of these formulae. With the help of MMT, the MMT Harvest is built by representing all theses theories (forumlae) explictly.

%Applying this principle to quantity expressions, we can consider different units as different theories and different quantity expressions different theorems belonging to their respective unit theories. The unit conversions can then be represented via views. The information on how to translate from one unit to another will be contained within the view. If we know all formulae in a corpus we can then generate an MWS Harvest that contains all the representations.



%As mentioned above, MathWebSearch is a search engine for forumlae and key phrases\ednote{Work in TEMA Search}. MWS consists of 3 main components as well as a frontend\footnote{The frontend is not part of MWS directly but rather built on top of the REST API, more on this later. }\cite{KohPro:MWSmanual}.

%The backend consists of 3 main components,
%\begin{enumerate}
%  \item a crawler,
%  \item a core system and
%  \item a public REST API.
%\end{enumerate}

%The crawler, as its name suggests, crawls corpera for forumlae. For each corpus MWS uses, a seperate crawler has to be implemented. The crawled formulae are passed to the core system and indexed in an MWS Harvest. The core system is also responsible for parsing queries and sending results back to the REST API. This is done by searching the harvest only. In order to make a semantic search for quantity expressions we will adapt this crawler to find and harvest quantity expression instead.

%Because MWS is semantics-aware, the harvest can not only contain the exact formulars that are found in the original corpus but also all versions that are quivalent to it. These are generated with the help of MMT and theory graphs, see sections \ref{sec:mws:mmt} and \ref{sec:mws:tg}.

%The frontend for MathWebSearch, which is not part of MWS itself but running client-side in a web browser, is written in HTML5, CSS and JavaScript. It accesses the REST backend and depends on MathML support to render Mathematics. When the client enters a query to search for, the \LaTeX{}ml daemon \cite{latexml-daemon} is used to transform the query into content MathML. Next, the client renders the MathML (to show the formular the user is searching for) and then sends the query off to the MWS API. Upon receiving results, the client renders them and links to the original documents. \ednote{ScreenShot of MWS} \ednote{Write about Query Expansion vs Index Expansion}

%There are several implementations of frontends and crawlers as well as extensions of MathWebSearch. One particular implementation is capable of crawling the arXMLiv corpus, which contains approximatly 750.000 documents. A list of demos can be found at \cite{URL:MWSDemo}.

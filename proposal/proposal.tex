\documentclass[11pt]{article}

% Input encoding
\usepackage[utf8]{inputenc}

% Math packages
\usepackage{amsfonts}
\usepackage{amssymb}
\usepackage{amsmath}
\usepackage{amsthm}

%Add some color
\usepackage{xcolor}

% WE want colored links instead of ugly boxes
\usepackage[colorlinks=true]{hyperref}

% Ednotes
\usepackage[show]{ed}

% We're actually using a units package, so we might also search this document.
\usepackage{siunitx}
\sisetup{load-configurations = abbreviations}

% BibTex
\usepackage{cite}

\title{Semantic Search for Quantity Expressions\\ \vspace{2 mm} Guided Research Proposal}
\author{Tom Wiesing\\Supervisor: Michael Kohlhase\\Jacobs University, Bremen, Germany}

\date{\today}

\begin{document}

%Title Page
\maketitle

%The abstract
\begin{abstract}
  In this proposal we describe how to approach Semantic search for Quantity Expressions, in particular how to make the existing MathWebSearch system aware of physical units.
  \ednote{Continue and finish abstract}
\end{abstract}

%The content pages
\section{Introduction}

In this paper we want to give an approach to Semantic Search for Quantity Expressions. A quantity expression is a number together with a physical unit, for example $25 \frac{\text{m}}{\text{s}}$ where $\frac{\text{m}}{\text{s}}$ the unit. This quantity expression is equivalent to $90 \frac{\text{km}}{\text{h}}$ (with equivalent in this sense meaning it expresses the same quantity) although the units are different. In a semantic search for quantity expressions, we want to be able to search for a certain quantity expression and find any equivalent ones as well.

MathWebSearch (MWS for short) is an existing semantics-aware system to search \LaTeX documents \footnote{Technically, MWS itself can only search XHTML documents. However with the help of \LaTeX ML it mostly handles converted \LaTeX  documents. } for mathematical formulae \cite{HamKohPro:man14}. As it is semantics-aware it not only searches for formulae in a simple-minded ``text search'' way but also includes simple transformation rules, such as $a + b = b + a$. Additionally, it can deal with wildcards such as $ { \color{red} x} + \sqrt{\color{red} x}$. For this query MWS would deliver formulars which are of the form as above with $ { \color{red} x} $ substituted with any sub-formular.

So far the transformation system has been used by MWS only for mathmatical forumlae. In this paper we propose an extension for quantity expressions which will be useful in physical papers.
The end-user will search, for example, \SI{100}{\degreeCelsius} and also get results which show $212$$^\circ$F or $373.15$K.

This proposal is organised as follows\ednote{Update this if we change the structure}: In section \ref{sec:mws} we shortly describe and discuss the existing MathWebSearch system and then proceed in section \ref{sec:extension} to describe in detail the proposed extension. Finally in section \ref{sec:problems_relatedwork} we discuss possible problems with this approach and related work.

\section{Background - The existing MathWebSearch system}
\label{sec:mws}

As mentioned above, MathWebSearch is a search engine for mathmatical formulars in documents. It has a corpus of ??? documents \ednote{Get an estimate here} and is currently deployed and used by Zentralblatt Math \ednote{link to their front-end here; mayber mention a bit more about it, this is thema-search}.

The frontend, running client-side in a web browser, is written in HTML5, CSS and JavaScript. It accesses a REST backend and dependens on MathML support to render Mathematics. It simply accessing a REST backend via AJAX. When the client enters a \LaTeX forumar to search for, the backend renders MathML which is then sent back to the client. Next, the client renders the MathML (to show the formular the user is searching for) and also sends back the MathML to the server to search for it. Finally the server sends back results to the client which then shows a list of them.

The backend, written in ??? \ednote{What is the backend written in?}, has 2 independent components. The first component, \LaTeX to MathML translation, is not part of MathWebSearch directly. It is rather uses \LaTeX ML to work\ednote{Really? Reference needed. }. The searching however is handled by MWS directly.

\ednote{Explain how the backend works}

\ednote{Find advantages of the current MWS system here}

Disadvantages of the current approach
\begin{itemize}
  \item has to be re-generated each time a document is added
\end{itemize}

\section{The proposed extension}
\label{sec:extension}

The goal of the guided research is to get a complete system that searches a corpus of documents for units as already indicated above. This system should be nicely accessible to the end user. Furthermore, it should be extendable with respect to
\begin{itemize}
  \item \textit{the unit system, } Adding new units should be easy and most translations should be deducted by the system automatically.
  \item \textit{and the searchable corpus. } It should be easy to search a different corpus of documents provided units are properly marked up inside it.
\end{itemize}

As with the existing system, the frontend should be a web page that works in all modern browsers. It should be mobile friendly. It should communicate directly with the backend via a RESTful API. The frontend should incoperate 4 main elements:

\begin{enumerate}
  \item a search input for a (real) quantity,
  \item a search inout for a unit, described further in section \ref{sec:problems_relatedwork},
  \item an additional search input for text and
  \item a result page that displays results and links to the found documents.
\end{enumerate}

\noindent The REST-based backend should have 3 major tasks:

\begin{enumerate}
  \item translate raw unit input into a standardised form (see section \ref{sec:problems_relatedwork} for details),
  \item search for documents using the input from the frontend and
  \item make the documents available to the enduser.
\end{enumerate}

\noindent The corpus \ednote{Write form here onwards}
\begin{itemize}
  \item should consist of a lot of tex documents
  \item should have marked up units
  \item ideally, if a single document is added, only the new corpus should have to be re-scanned (procedular generation)
  \item should be easily exchangable
\end{itemize}

The unit transition system \ednote{Write this part. }
\begin{itemize}
  \item should be a graph
  \item should have few connected components and each of the components should be sparse (i. e. few connections)
  \item translation are:
  \begin{itemize}
    \item either a factor towards a single unit
    \item or a composition of a factor together with a product or fraction of units \ednote{Figure out more details about this}
    \item Perhaps include prefixes somehow?
  \end{itemize}
\end{itemize}

\section{Problems and related Work}
\label{sec:problems_relatedwork}

There are several problems with this approach. \ednote{Continue intro pargraph}

The units have to be entered in some fashion in the search engine. While it is trivial to design an interface where a single unit can be entered, it is non-trivial when we want to recognise composite units as well. Furthermore si-prefixes (such as kilo or mini) should also be recognised which is an additional problem. There are a few alternatives to use as input formats: AsciiMath, \LaTeX and MathML, to name only 3. Independent of the input format, the end result (delivered to the backend of the search engine) should be MathML containing the unit in some yet-to-be-determined standard form.

The unit equivalences should be in the form of a theory graph \ednote{Reference this properly. } The equivalence
(translation) itself contains both a translation for the quantity and the unit. The values searched for should be simple real numbers together with the forumlar. However since all numbers must be represented in a finite fashion, translations formulars can gvie problems. A natural quantity in one unit can be a repeating decimal in a another unit. Also rounding of quantities might depend on the unit used. It is thus preferable not not search for an exact value but instead for a range of values. These ranges should depend on the unit. These ranges have just been implemented in MWS by Radu \ednote{Quote needed; reformulate}. \ednote{Write excatly how to use ranges} \ednote{Check if the reference in (2) is clear}

Finally, there are the problems of finding a corpus and marking up units in this corpus. The second problem, marking up units in a corpus, will be taken on by ???\ednote{Who is doing unit finding?} in a seperate thesis. The first problem will be postponed for now. In case there is still time, a suitably corpus can be marked up manually and plugged into the system. For first testing, I will create a dummy corpus of lorem-ipsum style documents which will contain a random sampling of units.

\ednote{Write final paragraph}

%BIBLIOGRAPHY
\bibliography{kwarc}{}
\bibliographystyle{plain}
\end{document}

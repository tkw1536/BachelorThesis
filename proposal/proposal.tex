\documentclass[11pt]{article}

% Input encoding
\usepackage[utf8]{inputenc}

% Math packages
\usepackage{amsfonts}
\usepackage{amssymb}
\usepackage{amsmath}
\usepackage{amsthm}

%Add some color
\usepackage{xcolor}

% WE want colored links instead of ugly boxes
\usepackage[colorlinks=true]{hyperref}

% Ednotes, TODO: Remove this in the end.
\usepackage[show]{ed}

% We're actually using a units package, so we might also search this document.
\usepackage{siunitx}
\sisetup{load-configurations = abbreviations}

% BibTex, TODO: Once we add references, we can uncomment this.
%\usepackage[backend=biber]{biblatex}
%\addbibresource{proposal.bib}


\title{Units for MathWebSearch\ednote{Preliminary Title}\\ \vspace{2 mm} Guided Research Proposal}
\author{Tom Wiesing\\Supervisor: Michael Kohlhase\\Jacobs University, Bremen, Germany}

\date{\today}

\begin{document}

%Title Page
\maketitle

%The abstract
\begin{abstract}
  In this proposal we describe an approach to introduce Units to MathWebSearch
  \ednote{Write abstract properly}
\end{abstract}

%The content pages
\section{Introduction}

\ednote{Write an introductary sentence / paragraph?}

MathWebSearch (MWS for short\ednote{should I really use abbreviations here?}) is a system to search (latex) documents for mathematical formulae. Additionally it can also search for text in the documents\ednote{Or is this the multi-faceted search that is currently planned? Do we really need this sentence?}. However it not only searches for formulae in a simple-minded string way but also includes simple transformation rules, such as $a + b = b + a$. Additionally, is is possible to search with wildcards such as $ { \color{red} x} + \sqrt{\color{red} x}$. In this example MWS delivers results of the given form where $\color{red} x$ is substituted with any sub-formular\ednote{Re-formulate this and link to an example}.

MWS has been shown to be very useful for mathematicians\ednote{Quote neeeded}. The transformation system it uses can currently be used only for mathematical formulae which limits its applications. In this paper we propose an extension for physical units\ednote{Reformulate this?}. Instead of transforming mathematical formulae, the search engine should transfer physical units.
The end-user will search, for example, \SI{100}{\degreeCelsius} and also get results which show $212$$^\circ$F or $373.15$K.

This proposal is organised as follows\ednote{Update this possible if we change the structure}: In section \ref{sec:mws} we describe the existing MathWebSearch system and then proceed in section \ref{sec:extension} to discuss the proposed extension. Finally in section \ref{sec:problems_relatedwork} we discuss possible problems with this approach and related work.

\section{The existing MathWebSearch system}
\label{sec:mws}

\ednote{Write this}

\section{The proposed extension}
\label{sec:extension}

\ednote{Write this}

\section{Problems and related Work}
\label{sec:problems_relatedwork}

\ednote{Write this}

%BIBLIOGRAPHY
%\printbibliography[title={\addtocounter{section}{1}\thesection \quad References}]
\end{document}

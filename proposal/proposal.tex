\documentclass[11pt]{article}

% Input encoding
\usepackage[utf8]{inputenc}

% Math packages
\usepackage{amsfonts}
\usepackage{amssymb}
\usepackage{amsmath}
\usepackage{amsthm}

%Add some color
\usepackage{xcolor}

% WE want colored links instead of ugly boxes
\usepackage[colorlinks=true]{hyperref}

% Ednotes, TODO: Remove this in the end.
\usepackage[show]{ed}

% We're actually using a units package, so we might also search this document.
\usepackage{siunitx}
\sisetup{load-configurations = abbreviations}

% BibTex, TODO: Once we add references, we can uncomment this.
%\usepackage[backend=biber]{biblatex}
%\addbibresource{proposal.bib}


\title{Units for MathWebSearch\ednote{Preliminary Title}\\ \vspace{2 mm} Guided Research Proposal}
\author{Tom Wiesing\\Supervisor: Michael Kohlhase\\Jacobs University, Bremen, Germany}

\date{\today}

\begin{document}

%Title Page
\maketitle

%The abstract
\begin{abstract}
  In this proposal we describe an approach to introduce Units to MathWebSearch
  \ednote{Write abstract properly}
\end{abstract}

%The content pages
\section{Introduction}

\ednote{Write an introductary sentence / paragraph?}

MathWebSearch (MWS for short\ednote{should I really use abbreviations here?}) is a system to search (latex) documents for mathematical formulae. Additionally it can also search for text in the documents\ednote{Or is this the multi-faceted search that is currently planned? Do we really need this sentence?}. However it not only searches for formulae in a simple-minded string way but also includes simple transformation rules, such as $a + b = b + a$. Additionally, is is possible to search with wildcards such as $ { \color{red} x} + \sqrt{\color{red} x}$. In this example MWS delivers results of the given form where $\color{red} x$ is substituted with any sub-formular\ednote{Re-formulate this and link to an example}.

MWS has been shown to be very useful for mathematicians\ednote{Quote neeeded}. The transformation system it uses can currently be used only for mathematical formulae which limits its applications. In this paper we propose an extension for physical units\ednote{Reformulate this?}. Instead of transforming mathematical formulae, the search engine should transfer physical units.
The end-user will search, for example, \SI{100}{\degreeCelsius} and also get results which show $212$$^\circ$F or $373.15$K.

This proposal is organised as follows\ednote{Update this possible if we change the structure}: In section \ref{sec:mws} we describe the existing MathWebSearch system and then proceed in section \ref{sec:extension} to describe in detail the proposed extension. Finally in section \ref{sec:problems_relatedwork} we discuss possible problems with this approach and related work.

\section{The existing MathWebSearch system}
\label{sec:mws}

\ednote{Write this}

\section{The proposed extension}
\label{sec:extension}

\ednote{Perhaps go over MWS again? Or just in the introduction?}

\begin{itemize}
  \item want a complete system
  \item searches a corpus of documents for units
  \item which is presentable to the end user
  \item should be extendable with respect to
  \begin{itemize}
    \item the corpus. Plugging in a new corpus should be as easy as running a script somehwere.
    \item the units. Adding new units should be simple by just adding a conversion to one already known unit.
  \end{itemize}
\end{itemize}

The frontend
\begin{itemize}
  \item a web page
  \item should work in modern browsers, preferably mobile-friendly
  \item should only be a frontend for a REST backend
  \item has an input for a unit
  \item has an input for a value
  \item maybe have facetet search on top
\end{itemize}

The backend
\begin{itemize}
  \item REST based
  \item based on the existing system
  \item has to have a format of units
  \item has to receive text queries
  \item has to receive exact values or ranges or automatically generated ranges
\end{itemize}

The corpus
\begin{itemize}
  \item should consist of a lot of tex documents
  \item should have marked up units
  \item ideally, if a single document is added, only the new corpus should have to be re-scanned (procedular generation)
  \item should be easily exchangable
\end{itemize}

The unit transition system
\begin{itemize}
  \item should be a graph
  \item should have few connected components and each of the components should be sparse (i. e. few connections)
  \item translation are:
  \begin{itemize}
    \item either a factor towards a single unit
    \item or a composition of a factor together with a product or fraction of units \ednote{Figure out more details about this}
    \item Perhaps include prefixes somehow?
  \end{itemize}
\end{itemize}

\ednote{Write this}

\section{Problems and related Work}
\label{sec:problems_relatedwork}

The unit input system:
\begin{itemize}
  \item Entering a single unit and recognising it is simple
  \item It is not clear how to enter composite units
  \item the end result delivered to the search engine should either be LaTeX or MathML
  \item maybe allow different inout methods:
  \begin{itemize}
    \item The output latex
    \item AsciiMath (with autocompletion would be nice)
    \item MathML?
  \end{itemize}
  \item system needs to be aware of full unit names as well as abbreviations
\end{itemize}

Unit translation
\begin{itemize}
  \item Should just be rational factors
  \item might give a problem with rounding
  \item maybe have ranges instead
  \begin{itemize}
    \item this has just been implemented by Radu \ednote{Quote needed}
  \end{itemize}
  \item support for composite units: $a \cdot{} b$ and $\frac{a}{b}$.
\end{itemize}

Finding a corpus
\begin{itemize}
  \item We need to have a suitably large corpus of documents to test this properly
  \item the units need to be marked up in the corpus
  \item actually finding them is done by ??? \ednote{Who is doing the unit finding?}
  \item The results should show which unit is originally in the text and also show the value in the unit searched for.
\end{itemize}

\ednote{Write this}

%BIBLIOGRAPHY
%\printbibliography[title={\addtocounter{section}{1}\thesection \quad References}]
\end{document}
